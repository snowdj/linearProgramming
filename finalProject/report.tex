\documentclass[twocolumn]{article}

\usepackage[sort&compress,numbers]{natbib}
\usepackage{hyperref}
\usepackage{url}
\usepackage{amsfonts}
\usepackage{amsmath}
\usepackage{multirow}


\title{Integer linear programming in routing problems}

\author{
    Kelly Kaoudis \\
    CSCI 5654, Fall 2013 \\
}
\date{\today}

\begin{document}
\maketitle

\section*{Introduction}

Routing problems, which are NP-complete, can be reduced to ILPs and iteratively solved in reasonable time.
Linear program standard form\cite{Vanderbei} is

$
\begin{aligned}
    max \; & \zeta = {\sum\limits^{n}_{j=1}c_j x_j} \\
    st \; w_i & = b_i - {\sum\limits^{n}_{j=1}a_{ij} x_j} \; where \, i = {1, ..., m} \\
          & x_j, w_i \geq 0 \\
\end{aligned}
$

We examine methods for \textit{reducing} to an ILP, but it
is worth mentioning that with heuristics such as interior point methods or the simplex method with
LP relaxation, branch-and-bound, and/or cuts including lift-and-project, optimal solutions for ILPs
are found in polynomial time. In routing problems, we \textit{minimize} to find the least interrupted
(and possibly even shortest) route from each start to endpoint, simultaneously. In an LP,
decision variables $x_j$ and problem-internals (``slack" variables, $ w_i $) can take any real values.
ILPs constrain decision and slack variables $x_j, w_j \in \mathbb{Z}$. In a binary ILP, decision variables
are further constrained: $0 \leq x_j \leq 1$, $\forall x_j \in \vec{x}$. Two problems which reduce to ILPs are the global routing approximation stage in laying out VLSI circuits,
and reducing rail line crossings in metropolitan public transportation route-maps.

\section*{Integrated circuit layout}

A minimal interconnection set with low route congestion, especially at nanometer scale, improves
circuit layout\cite{wu}.
Layout has four subproblems: partitioning, placement, global routing
approximation, and detailed local routing. In global routing, which is NP-hard, we
represent each net by a set of feasible rectilinear minimum Steiner trees (RMSTs),
then choose the set of trees (one per net in the netlist) to minimize interconnection.

Given a set of pins that should connect, a spanning tree\cite{Sherwani} built with only vertical
and horizontal line segments is a feasible route for that net.
For a net of $r$ cells, there are $r^{r-2}2^{r-1}$ constructable RMSTs with minimum vias,
so linear programming becomes a very attractive way to find the optimal tree set.

Our tree sets $N_k$ also help approximate routing demand per net.
Assign a $y_j \in \vec{y}$ to each tree $j$ with $y_j = 1$ if $j \, \in$ final layout,
0 otherwise. $\vec{y}$ is the set of \textit{all} trees for the netlist.
The probability that a $y \in N_k$ is the best tree for net $k$ is
$p(y_j) = \frac{1}{no. \, trees \in N_k}$. $\forall$ edges $e_i \in y_j$,
estimate the number of \textit{other} trees that pass through:
$r(e_i) = \sum\limits^{t}_{j=1}a_{ij}p(y_j)$, the routing demand for tree $j$.
$t$ is the total trees $\in \vec{y}$. $a_{ij} = 1$ if $j$ intersects $e_i$ and 0 otherwise.

The sum of routing demands of other trees' edges $y_j$ passes through, $c_j = \sum r(e_i)$ where $e_i \in y_j \in \vec{y}$, is produced once all trees are generated, and will be incorporated as a
penalty function on the ILP's objective $\forall$ trees $j$. For nets passing through congested edges,
construct additional trees with perhaps a greater wire length or number of pins, but avoiding those edges.

We also calculate $f_{congestion} = c - r(e_i)$ and
$f_{utility} = \frac{c - r_{l}(e_i)}{r_{u}(e_i) - r_{l}(e_i)}$,
the ratio between the ``available" vias for an edge after routing nets with
one or two possible trees, and the number of trees that could pass through the
currently available tracks. $c$ is the max via-capacity of an edge (channel).
If $f_{congestion}(e_i) < 0$ our estimation is greater than the max capacity of $e_i$, so routing
the circuit as-is is infeasible.
If $f_{congestion}(e_i) < \frac{c}{10}$, (channel is at more than 90 per cent
capacity), re-route nets with vias on $e_i$
until $e_i$ is no longer above the threshold and don't use it as a possibility
for more vias. Only consider the pre-ILP circuit feasible if edge-congestion is below
90 per cent $\forall$ edges in our selected trees.

\section*{Metro rail-map layout}

Designing a readable, reasonably geographically coherent, aesthetically pleasing metro map also has an
NP-hard\cite{argyriou} routing component. Asquith et al.\cite{asquith} were the first to
translate line-crossing (visual congestion) minimization to an integer linear program over an
unrestricted plane graph underlying the metro map where stations are vertices and edges are parts of
transit lines, which may cross at or between vertices. The ILP construction is a bit different from
\cite{behjat}, though parallels exist. Visual clarity in a metro map improves with fewer
crossings at vertices; however, transit route utility is greater when a route
has more shared stops. While calculation time for the optimal number of line-crossings in
a metro layout is polynomial when the map's terminal positions are predetermined\cite{nollenburg},
the general case is NP-complete.

Their algorithm requires four steps:
\begin{enumerate}
        \setlength{\itemsep}{0cm}
        \setlength{\parskip}{0cm}
        \belowdisplayskip=0pt
        \abovedisplayskip=0pt
    \item Calculate all maximal common sub-paths for each pair of individual routes (metro lines) $(g,h)$
        traversed in the network $\delta_{1}(g,h) ... \delta_{m}(g,h)$
    \item Convert each of these into a crossing rule $c$ dependent upon position of the terminals
        that tells whether $g$ and $h$ cross
    \item Formulate an ILP with the problem information to determine how many crossings exist in a minimal
        representation of the graph for the metro map
    \item Decide the ordering of lines at each vertex for aesthetics
\end{enumerate}

A crossing rule tells in what ordering a crossing occurs. The authors prove that in an optimal ILP solution,
two paths $g$ and $h$ cross exactly 1 or 0 times. Either
\begin{itemize}
        \setlength{\itemsep}{0cm}
        \setlength{\parskip}{0cm}
        \belowdisplayskip=0pt
        \abovedisplayskip=0pt
    \item path g is above path h at $v_0$ AND path h is above path g at $v_m$ or
    \item path h is above path g at $v_0$ AND path g is above path h at $v_m$
\end{itemize}
where $v_0$ and $v_m$ are the end vertices of the maximal common subpath for $(g,h)$. Each clause in the
previous crossing rule is a Boolean variable during construction of the ILP.

Reducing rail-route crossings in \cite{asquith} and reducing vias and edge-congestion
in \cite{behjat} appear to be very similar problems: both are NP-complete, NP-hard, and affect multiple
aspects of the rest of the problem. In both IC layout and map construction,
we lessen graph congestion as much as possible. We consider a via in circuit design and a rail line crossing
analogous here.

\section*{Problem representations}

We'll sketch the ILP formulations resulting from the two above procedures and briefly compare.

Each crossing rule $c$ in the metro graph problem is represented by a $w_i \in \vec{w}$ and constrained
such that its $w_i$ is 1 if there is a crossing and 0 otherwise.
Every terminus referenced in at least one $c_i \in \vec{C}$ is included in the ILP.
The metro-line ILP $\zeta$ is $min \; \sum\limits^{n}_{i=1}w_i$, subject to at least
one constraint on each $w_i$, dependent on the specific graph instance. The circuit problem is set up a bit
differently in that its objective takes into account more aspects of via reduction; one might even go so
far as to say the metro problem is akin to a smaller, perhaps more constrained instance of the
circuit problem, depending on the sizes and edge/route-congestion of the map and circuit in question.

In the global routing problem, Vannelli originally proposed an ILP for just minimizing total wire length
\cite{Vannelli}. Behjat et al.\cite{behjat} say this older model can
produce infeasible results on several of the largest MCNC\cite{mcnc} benchmark circuits.
Un-necessary congestion may remain in the optimized result because the ILP considers
incomplete information, which results in hotspots, capacitive coupling, and other instability. If channel
capacities are fixed at 90 per cent of their actual values, these issues are much less likely.
Their second previous attempt at ILP formulation reduces vias on a tree $j$ as well as in $j$'s path but
doesn't minimize channel capacities. The model they propose combines elements of the previous to
minimize the channel capacity and also wire length, number of vias, and congestion simultaneously.

$
\begin{aligned}
    min \; & \sum\limits^{t}_{j=1}w_{j}y_{j} + \beta_{z}z, \; where \, z \in {0, ..., c} \\
    st \; & \sum\limits_{y_{j} \in N_{k}}y_j = 1, \; k = {1, ..., n} \\
    & \sum\limits^{t}_{j=1}a_{ij}y_{j} - z \leq 0, \; i = {1, ..., p} \\
    & y_j \in {0,1}, \; where \, j = {1, ..., t} \\
\end{aligned}
    \belowdisplayskip=0pt
    \abovedisplayskip=0pt
    \setlength{\parskip}{0cm}
$

$w_j = \beta_{l}w_{l_{j}} + \beta_{v}w_{v_{j}} + \beta_{c}w_{c_{j}}$ where
$w_{v_{j}}$ is a weight on the number of vias, $w_{c_{j}}$ is a weight for congestion in the path
of tree $j$, and scalars $\beta_{l, v, c}$ change weighting factor emphasis. $z$ is the channel
capacity, and $\beta_z \in \mathbb{R}$ is the weight for max channel capacity.

While the metro-crossing ILP appears more constrained in channel capacity
(a line is crossed at most once by every other metro line) than the circuit
routing problem, if the metro map in question has enough intertwined lines,
the channel capacities of rail-line segments between vertices and RMST edges could
be comparable. If there are a great enough number of lines (or sub-circuits) in a
routing problem and we apply both ILP formulations, it may be possible that the IC
formulation, which requires all edges in the RMSTs in a feasible solution to have
90 per cent or less of their channel capacities fulfilled, could produce a more
minimal result than the metro ILP.

\section*{Further work}

Additional exploration into ILPs for routing with minimal congestion
should examine more works and also verify results.
It would be interesting to try the metro map ILP on MCNC
benchmarks of varying sizes, and also to fit Behjat et al.'s circuit ILP on metro
graphs for time-efficiency and optimization comparison. Due to space constraints,
not everything presented in \cite{behjat} and \cite{asquith} could be covered this time.
\bibliographystyle{named}
\bibliography{report}

\end{document}
